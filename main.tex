\documentclass{configuracion}
\usepackage{lipsum}

\begin{document}

%----------- Información del informe ---------

\titulo{Título del trabajo} %título del archivo .pdf | titulo ppal
\tema{Subtítulo del trabajo \\ \textbf{Temática del trabajo}} % Titulo 2

\profesor{Don. \textsc{Nombre}\\Apellido} %Nombre del profesor
\piepagina{Trabajo XX}

\estudiantes{Rubia López \textsc{Víctor José}} %Nombre del alumno
%----------- Inicialización -------------------
        
\crearlogos %Mostrar márgenes
\crearportada %Crear la portada
\crearindice %Creación del índice de contenidos

%------------ Cuerpo del informe ----------------


\section{Sección X} 


\input{}

\bibliographystyle{apacite} % We choose the "plain" reference style
\bibliography{refs} % Entries are in the refs.bib file


% %------------- Comandos útiles ----------------

% \section{Comandos útiles}

% Estos son algunos comandos útiles :

% %------ Para insertar y citar una imagen centralizada -----

% \insererfigure{logos/Logo_UGR.jpg}{3cm}{Leyenda}{Label de la figure}
% % El primer argumento es el camino hacia la imagen
% % La segunda es la altura de la foto
% % La tercera leyenda
% % El cuarto es la etiqueta para las referencias
% Aquí cito la imagen \ref{fig: Label de la figure}


% %------- Para insertar y citar una ecuación --------------

% \begin{equation} \label{eq: exemple}
% \rho + \Delta = 42
% \end{equation}

% La ecuación \ref{eq: exemple} se cita aquí. 

% % ------- Para escribir variables ----------------------

% Para escribir variables en el texto, basta con poner el símbolo \$ entre el texto deseado como: constante $\rho$. 

\end{document}
